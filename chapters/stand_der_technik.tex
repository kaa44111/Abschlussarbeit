\chapter{Literaturrecherche und Stand der Technik}\label{sec:stand_der_technik}

Es gibt zahlreiche Ratgeber\index{Ratgeber} für das wissenschaftliche Arbeiten und Schreiben. Die Handbücher unterscheiden sich in inhaltlichen Schwerpunkt, praktischer Orientierung und Vertiefung der einzelnen Themen. Drei sehr empfehlenswerte Ratgeber sollen kurz vorgestellt werden.

\cite{karmasin2012gestaltung} bieten einen sehr knappen und praktisch orientierten Ratgeber. Es werden inhaltliche und formale Anforderungen an wissenschaftliche Arbeiten wie inhaltlicher Aufbau der Kapitel, Bewertungskriterien und formale Aspekte wie Gliederung behandelt. Daneben enthält der Ratgeber ein eigenes Kapitel mit Tipps zur Formatierung mit Word.

Das Handbuch von \cite{esselborn2012richtig} konzentriert sich auf die Frage nach dem richtigen wissenschaftlichen Sprachstil. Es werden konkrete Regeln und Übungen vorgestellt um sprachliche Präzision und gedankliche Klarheit im Text zu erreichen. Daneben wird in einem eigenen Kapitel auf die häufigsten Fehler beim wissenschaftlichen Schreiben hingewiesen.

\cite{balzert2011wissenschaftliches} bieten einen sehr ausführlichen Ratgeber zum wissenschaftlichen Arbeiten. Im ersten Teil werden Qualitätskriterien und Methoden als Grundlagen wissenschaftlicher Arbeit aufgezeigt. Im zweiten Teil werden verschiedene wissenschaftliche Artefakte also Textformen gegenübergestellt und der formale Aufbau wissenschaftlicher Arbeiten beleuchtet. Im dritten Teil werden Empfehlungen zum Erstellungsprozess einer Arbeit mit Projektplan etc. gegeben. Im letzten Teil werden verschiedene Aspekte der Präsentation behandelt, wie z. B. Vortragsformen mit und ohne visuelle Unterstützung oder der richtige Vortragsstil.

\section{Überblick über die Industrielle Bildverarbeitung}

\section{Deep Learning in der Bildverarbeitung}

\section{Aktuelle Forschung und Entwicklungen}

