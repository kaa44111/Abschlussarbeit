\PassOptionsToPackage{main=ngerman}{babel}
%\PassOptionsToPackage{main=english}{babel}

\documentclass[thesis]{mi-document}

\bachelor % im Falle einer Masterarbeit \master

% Variablen, die für das Deckblatt und Metadaten verwendet werden
\title{Evaluierung von Einsatzmöglichkeiten einer Deep Learning Applikation zur Detektion gelabelter Merkmale in Bildern}
\author{Amina Kasa}
\semester{Sommersemester 2024}
\course{Bachelorarbeit}
\module{Fakultät Informatik und Mathematik}
\dozent{Prof. Dr. Timo Baumann}
\studid{3190755}
\studSemester{Medizinische Informatik}
\phone{0941/133742666} % Optional
\studSubject{Medizinische Informatik}
\firstReviewer{Prof. Dr. Timo Baumann}
\secondReviewer{[Zweitgutachter]}
\advisor{GEFASOFT Automatisierung und Software GmbH}
\address{Klenzestr. 2a, 93051 Regensburg}{} % Optional
\mail{kasaamina@gmail.com}
\studMail{amina.kasa@st.oth-regensburg.de}
\dateHandedIn{[Abgabetermin der Arbeit]}
\keywords{Enter;key;words;here}

\bibliographystyle{apalike}
% Falls Sie die Abkürzung zum Einbinden von Grafiken benutzen möchten. Erläuterung fnden Sie im Abschnitt zu Abbildungen.
\input{misc/config}

\usepackage{emptypage}
\pagenumbering{roman}

\begin{document}
% ---------------------------------------------------------------
\pagestyle{empty}
% Das Deckblatt erstellen
 

% \newlength\headpicwd
% \setlength{\headpicwd}{0.2\pdfpagewidth} % image width of physical page?
% \newcommand\printlogo{%
%   \AddToShipoutPictureBG*{%
%     \AtStockUpperLeft{%
%       \raisebox{-2cm}{%
%       \hspace{2.5cm}
%         % centred with respect to physical page ..?
%         % \hspace*{\dimexpr 0.5\pdfpagewidth - 0.5\headpicwd\relax}%
%         \includegraphics[keepaspectratio=true, width=0.33\pdfpagewidth]{images/template/unili_logo.jpg}%
%       }%
%     }%
%   }%
% }

\begin{titlepage}
\setstretch{1.5}

\vspace*{-30pt}
\includegraphics[width=0.5\textwidth]{images/Unbenannt}
\hfill\includegraphics[width=0.5\textwidth]{images/logo_firma}
\vspace{20pt}

\centering
\Large
\setstretch{1.5}
\textbf{\getInstitute}\\
\large
\textbf{\getModule}\\
\normalsize
\textbf{Studiengang \getStudSubject}

\vspace{60pt}
\setstretch{1.5}
\huge
\textbf{Exposé zur}

%\vspace{5pt}
\setstretch{1.5}
\Huge
\textbf{\getWorkType}

\normalsize

\textmd{\getAuthor}
\vspace{20pt}

\Large
\textbf{\getTitle}

\vspace{20pt}

\normalsize


%\ifcurrentbaselanguage{German}{
%\getWorkType im \getStudSubject{}\\
%at \getInstitute
%}

%\ifcurrentbaselanguage{English}{
%\getWorkType in \getStudSubject{}\\
%at \getInstitute
%}

\setstretch{1.5}

\vfill

\begin{tabularx}{\linewidth}{@{}l<{:}>{\RaggedRight\arraybackslash}X@{}}
\ifcurrentbaselanguage{German}{

Vorgelegt von&\getAuthor\\
Matrikelnummer&\getStudID\\
E-Mail &\getStudMail\\
Erstgutachter&\getFirstReviewer\\
Zweitgutachter&\getSecondReviewer\\
Betreuende Firma&\getAdvisor\\
Abgegeben am&\getDateHandedIn

}

\end{tabularx}
\end{titlepage}


% \thispagestyle{empty}
% \quad \addtocounter{page}{-1}
% \newpage
% \pagenumbering{roman}   


 \newpage

 % Abstract hinzufügen
%    \clearpage
%    \doublespacing

%    \input{expose/0-Zusammenfassung}
%    \clearpage 

% ---------------------------------------------------------------
%Inhaltsverzeichnisse  
\pagestyle{plain}
\singlespacing
\addcontentsline{toc}{chapter}{Inhaltsverzeichnis}
\tableofcontents % Optional
\newpage

% \addcontentsline{toc}{chapter}{Abbildungsverzeichnis}
% \listoffigures % Optional
% \newpage

% \addcontentsline{toc}{chapter}{Tabellenverzeichnis}
% \listoftables % Optional
% \newpage

% \addcontentsline{toc}{chapter}{Quellcodeverzeichnis}
% \lstlistoflistings % Optional
% \newpage
% --------------------------------------------------------------
\pagenumbering{arabic}
\pagestyle{headings} % Seitennummern und Kapitelbezeichnungen anzeigen
    
    % hier beginnt der eigentliche Inhalt der Arbeit
    \chapter{Einleitung}\label{sec:exp_einleitung}

    \chapter{Elevator Pitch}\label{sec:exp_elevator}
Meine Bachelorarbeit untersucht, wie die U-Net-Architektur trotz begrenzter Trainingsdaten für die semantische Segmentierung in stark verrauschten industriellen Bildern angepasst und optimiert werden kann. Ziel ist es, die Effizienz und Genauigkeit dieser Architektur bei verschiedenen Datensätzen zu überprüfen und evaluieren.

    \chapter{Forschungsfragen und Zielsetzung}\label{sec:exp_ziel}

\section{Forschungsfragen}

\begin{enumerate}

\item Wie effektiv kann das U-Net-Modell mittels semantischer Segmentierung gelabelte Merkmale in stark verrauschten industriellen Bildern detektieren, und welche Genauigkeit lässt sich dabei erreichen?

\item Welche Methoden der Datenvorbereitung (z.B. Datenaugmentation, Bildvorverarbeitung) und der Modellanpassung (z.B. Architekturmodifikationen, Hyperparameter Tuning ) können die Leistungsfähigkeit des U-Net-Modells unter den Bedingungen von starkem Bildrauschen und begrenzter Datenmenge verbessern?

\end{enumerate}

\section{Zielsetzung}

Das Ziel dieser Bachelorarbeit ist es, die Erkennungsgenauigkeit von Objekten in industriellen Bildern mithilfe eines leicht modifizerten U-Net-Modells zu maximieren. Dabei werden insbesondere Herausforderungen wie starkes Bildrauschen und begrenzte Datenmengen adressiert.
Zur Zielerreichung wird die U-Net-Architektur angepasst und optimiert, indem die spezifischen Eigenschaften der Bilddaten analysiert und die Hyperparameter abgestimmt werden. PyTorch dient als Implementierungs-Framework. Eine umfassende Literaturrecherche wird den aktuellen Stand der Technik im Bereich der \gls{dl}-basierten \gls{od} beleuchten und aktuelle Trends, Herausforderungen sowie Lösungsansätze identifizieren. Abschließend erfolgt eine systematische Evaluierung der Modellleistung, um die Genauigkeit und Effizienz der Objektdetektion zu bewerten. Dies soll zu optimalen Strategien für die industrielle Qualitätskontrolle beitragen.

    \chapter{Methodik}\label{sec:exp_methodik}
\begin{enumerate}
    \item Datenbeschaffung und -vorbereitung:
    
    \begin{itemize}
        \item Datensätze: Verwendung von industriellen Bilddatensätzen mit relevanten Objekten und Defekten aus dem Betrieb. Es werden konkrete Beispiele verwendet, die in der Praxis nicht mittels traditioneller Bildverarbeitung gelöst werden könnten.
        \item Labeling: Die Bereiche der Bilder die den Fehler zeigen, werden für jeden Bild händisch aufgezeichnet.
    \end{itemize}
    
    \item Modellimplementierung:
    \begin{itemize}
        \item Software: Einsatz von Python mit Bibliotheken wie Pytorch
        \item Modellanpassung: Anpassung der U-Net-Architektur, z. B. durch Modifizierung der verschiedenen Layers um die effizienz und geschwindigkeit des trainier vorgangs zu verbessern.
        \item Hyperparameter-Tuning: Systematische Optimierung von Lernrate, Batch-Größe und anderen Parametern
    \end{itemize}
    
    \item Training und Validierung::
    \begin{itemize}
        \item Trainingsstrategie:  Verwendung von K-Fold-Cross-Validation.
        \item Datenaugmentation: Einsatz von Techniken wie Rotation, Spiegelung, Skalierung und Helligkeitsvariation.
        \item Regularisierung: Anwendung von Methoden wie Dropout oder Early Stopping.
    \end{itemize}

    \item Evaluation:
    \begin{itemize}
        \item Metriken: Bewertung anhand von Metriken wie Intersection over Union (IoU), Dice-Koeffizient, Präzision, Recall und F1-Score.
        \item Analyse der Rauschresistenz: Untersuchung der Modellleistung bei unterschiedlichen Rauschpegeln und -arten.
    \end{itemize}
    
    \item Analyse und Diskussion:
    \begin{itemize}
        \item Ergebnisinterpretation: Identifikation von Erfolgsfaktoren und Limitierungen.
        \item Praktische Implikationen: Bewertung der Anwendbarkeit in industriellen Szenarien.
        \item Empfehlungen: Entwicklung von Leitlinien für die Implementierung.
    \end{itemize}
    
\end{enumerate}
    \chapter{Vorläufige Gliederung}\label{sec:exp_gliederung}
Die vorliegende Arbeit beginnt mit einer Einführung in das Thema, welche in ihrer Struktur und ihrem Inhalt der Einleitung des ersten und zweiten Kapitels des vorliegenden Exposés ähnelt.
In der Folge erfolgt eine systematische Erschließung der relevanten Literatur. Dabei werden grundlegende Konzepte präsentiert. Die für die Arbeit zentralen Begriffe wie "industrielle Bildverarbeitung" und "Deep Learning" werden in diesem Kontext näher erläutert und in ihrer Bedeutung für die vorliegende Untersuchung herausgearbeitet. In diesem Kontext werden aktuelle Forschungsergebnisse sowie der Stand der Technik in den relevanten Bereichen erörtert.
Im Folgenden wird die angewandte Methodik detailliert beschrieben, wie bereits in Abschnitt \ref{sec:exp_methodik} ausführlich erläutert. Dies umfasst die Präparierung der Bilddaten, die Modellierung und das Training des U-Net Modells sowie die Validierung und Bewertung der Ergebnisse.
Abschließend werden die Ergebnisse zusammengefasst und im Hinblick auf die Forschungsfragen diskutiert. Ein Ausblick auf zukünftige Arbeiten rundet die Arbeit ab und bietet Perspektiven für weiterführende Forschung.

\section{Visualisierte Darstellung}
% Provide an expected Table of Content of the final thesis.
\centering
\preliminaryTOCbox{
% You can remove the "box" by deleting the \preliminaryTOCbox command
\begin{enumerate}[label=\arabic*,font=\bfseries] 
    \item[] \hspace{-1cm} \textbf{Abstract}\enspace  \secdotfill \enspace i
    %\item \textbf{Introduction} \secdotfill 1
    \item \textbf{Einleitung} \secdotfill  i
        \begin{enumerate}[label*=.\arabic*]
            \item Hintergrund und Motivation \subsecdotfill i
            \item Zielsetzung und Fragestellung \subsecdotfill i
            \item Aufbau der Arbeit \subsecdotfill i
        \end{enumerate}
        
    \item \textbf{Theoretische Grundlagen und Stand der Technik} \secdotfill  i
        \begin{enumerate}[label*=.\arabic*]
        \item Grundlagen der Objektdetektion \subsecdotfill i
            \begin{enumerate}[label*=.\arabic*]
            \item Industrielle Bildverarbeitung \subsecdotfill i
            \item Deep Learning \subsecdotfill i
                \begin{enumerate}[label*=.\arabic*]
                \item Supervised Learning \subsecdotfill i
                \item Unsupervised Learning \subsecdotfill i
                \end{enumerate}
            \end{enumerate}
        \item Aktuelle Forschung und Entwicklungen \subsecdotfill i
        \end{enumerate}
    
    \item \textbf{Implementierung und Experimente} \secdotfill  i
    \begin{enumerate}[label*=.\arabic*]
        \item Präparierung der Bilddaten \secdotfill  i
        
        \begin{enumerate}[label*=.\arabic*]
            \item Vorbereitung und Beschreibung des Datensatzes \subsecdotfill i
            \item Labeling-Prozess \subsecdotfill i
            \item Datenaugmentationstechniken \subsecdotfill i
        \end{enumerate}
        
        \item Modellierung und Training \secdotfill  i
        \begin{enumerate}[label*=.\arabic*]
            \item Auswahl der Softwarebibliotheken \subsecdotfill i
            \item Modellarchitektur \subsecdotfill i
            \item Trainingsprozess und Hyperparameter-Optimierung \subsecdotfill i
        \end{enumerate}
    
        \item Validierung und Bewertung der Ergebnisse \secdotfill i
        \begin{enumerate}[label*=.\arabic*]
            \item  Methoden der Validierung \subsecdotfill i
            \item Bewertungsmethoden und Metriken \subsecdotfill i
            \item Analyse und Diskussion der Ergebnisse \subsecdotfill i
        \end{enumerate}
    \end{enumerate}

    \item \textbf{Fazit und Ausblick} \secdotfill  i
        \begin{enumerate}[label*=.\arabic*]
            \item  Zusammenfassung der Ergebnisse \subsecdotfill i
            \item Ausblick auf Zukünftige Arbeiten \subsecdotfill i
        \end{enumerate}
    
    \preliminaryTOCend
 
\end{enumerate}
}
    \chapter{Zeitlicher Ablauf}\label{sec:exp_ablauf}
\begin{table}[h!]
    \centering
    \begin{tabular}{|p{3cm}|p{3cm}|p{8cm}|}
        \hline
        \textbf{Phase} & \textbf{Zeitraum} & \textbf{Beschreibung} \\
        \hline
        \textbf{Woche 1 – 2} & Literaturrecherche & 
        - Einarbeitung in den Stand der Technik \newline
        - Lesen und Zusammenfassen relevanter Studien zu U-Net und Bildverarbeitung \\
        \hline
        \textbf{Woche 3 – 4} & Datenpräparierung & 
        - Sammeln und Annotieren von Bilddaten \newline
        - Durchführung von Datenaugmentation mit Bildverarbeitungssoftware \\
        \hline
        \textbf{Woche 5 – 6} & Modellimplementierung & 
        - Auswahl geeigneter Softwarebibliotheken, wie \textit{PyTorch} \newline
        - Implementierung der U-Net-Architektur \\
        \hline
        \textbf{Woche 7 – 8} & Modelltraining und -anpassung & 
        - Training des U-Net-Modells auf den vorbereiteten Datensätzen \newline
        - Optimierung durch Hyperparameter-Tuning und Modellanpassung \\
        \hline
        \textbf{Woche 9-10} & Validierung & 
        - Evaluierung der Modellleistung durch Metriken wie Genauigkeit, Präzision und Recall \\
        \hline
        \textbf{Woche 11} & Ergebnisanalyse & 
        - Analyse und Interpretation der Ergebnisse \newline
        - Diskussion der Stärken und Schwächen der Methode \\
        \hline
        \textbf{Woche 12} & Fertigstellung und Abgabe & 
        - Abschluss der Bachelorarbeit \newline
        - Einholen von Feedback \newline
        - Letzte Überarbeitungen und Abgabe der Arbeit \\
        \hline
    \end{tabular}
    \caption{Zeitlicher Ablauf der Bachelorarbeit}
\end{table}
    
    % Kapitelbezeichnung in der rechten oberen Ecke entfernen
\clearpage
\pagestyle{plain}

% --------------------------------------------------------------   
% Literaturverzeichnis anzeigen
% kleinerer Zeilenabstand, damit es nicht so gestreckt aussieht
\onehalfspacing
\bibliography{references}
\doublespacing
\end{document}
