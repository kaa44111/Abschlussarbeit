\chapter{Forschungsfragen und Zielsetzung}\label{sec:exp_ziel}
\section{Forschungsfragen}
\begin{enumerate}
    \item Wie effektiv kann das U-Net-Modell gelabelte Merkmale in stark verrauschten industriellen Bildern mittels semantischer Segmentierung detektieren, und welche Genauigkeit kann dabei erreicht werden?
    \item Welche Techniken der Datenvorbereitung (z. B. Downsampling, Datenaugmentation) und Modellanpassung (z. B. Architekturmodifikationen, Hyperparameter-Tuning) können die Leistungsfähigkeit des U-Net-Modells unter Rauscheinflüssen verbessern?
\end{enumerate}

\section{Zielsetzung}
DCNNs proved effective in semantic segmentation, following a three-phase procedure: preprocessing, processing, and output generation.\cite{manakitsa_review_2024}
Das Ziel dieser Bachelorarbeit ist es, die Segmentierung von Merkmalen in Bildern mithilfe einer U-Net-Architektur zu automatisieren. Es werden segmentierte Bilder verwendent, um Defekte oder relevante Objekte zu erkennen und ihre Grenzen präzise zu bestimmen. Die U-Net-Architektur ist ein universeller Funktionsapproximator mit hoher Genauigkeit und eignet sich besonders gut für verrauschte Eingabedaten (Chen et al., 2016). Ursprünglich für die biomedizinische Bildsegmentierung entwickelt, könnte U-Net auch in industriellen Anwendungen von großem Nutzen sein, insbesondere beim Umgang mit komplexen Bilddaten. Das Ziel meiner Arbeit ist es daher, diese Methode zu untersuchen, um die Analyse dieser Bilddatensätze zu verbessern und gleichzeitig zeitaufwendige manuelle Prozesse zu minimieren oder zu vermeiden.


Das Ziel dieser Bachelorarbeit ist es, die U-Net-Architektur für die semantische Segmentierung in industriellen Bildern zu untersuchen und zu optimieren. Ein zentraler Bestandteil der Arbeit ist die Einarbeitung in den Stand der Technik durch eine umfassende Literaturrecherche, um aktuelle Trends und Herausforderungen im Bereich der Deep-Learning-basierten Bildverarbeitung zu verstehen.

Darüber hinaus wird ein besonderer Fokus auf die Präparierung der Bilddaten gelegt, indem Bilddatensätze sorgfältig annotiert und durch Techniken wie Datenaugmentation erweitert werden. Hierbei kommt verfügbare Bildverarbeitungssoftware zum Einsatz, um die Qualität der Trainingsdaten zu maximieren.

Ein weiterer Schritt ist die Auswahl geeigneter Softwarebibliotheken, wobei die Implementierung des Modells in PyTorch vorgenommen wird. Die optimale Modellbildung und Anpassung an spezifische Anforderungen der industriellen Bildverarbeitung stehen im Vordergrund.

Abschließend erfolgt die Validierung des Modells durch eine systematische Bewertung der Ergebnisse. Die Arbeit zielt darauf ab, die Effizienz und Genauigkeit des U-Net-Modells zu bewerten und es mit traditionellen Methoden der Bildverarbeitung zu vergleichen, um die besten Strategien für die industrielle Qualitätskontrolle zu identifizieren.
