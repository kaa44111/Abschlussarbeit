\chapter{Forschungsfragen und Zielsetzung}\label{sec:exp_ziel}

\section{Forschungsfragen}

\begin{enumerate}

\item Wie effektiv kann das U-Net-Modell mittels semantischer Segmentierung gelabelte Merkmale in stark verrauschten industriellen Bildern detektieren, und welche Genauigkeit lässt sich dabei erreichen?

\item Welche Methoden der Datenvorbereitung (z.B. Datenaugmentation, Bildvorverarbeitung) und der Modellanpassung (z.B. Architekturmodifikationen, Hyperparameter Tuning ) können die Leistungsfähigkeit des U-Net-Modells unter den Bedingungen von starkem Bildrauschen und begrenzter Datenmenge verbessern?

\end{enumerate}

\section{Zielsetzung}

Das Ziel dieser Bachelorarbeit ist es, die Erkennungsgenauigkeit von Objekten in industriellen Bildern mithilfe eines leicht modifizerten U-Net-Modells zu maximieren. Dabei werden insbesondere Herausforderungen wie starkes Bildrauschen und begrenzte Datenmengen adressiert.
Zur Zielerreichung wird die U-Net-Architektur angepasst und optimiert, indem die spezifischen Eigenschaften der Bilddaten analysiert und die Hyperparameter abgestimmt werden. PyTorch dient als Implementierungs-Framework. Eine umfassende Literaturrecherche wird den aktuellen Stand der Technik im Bereich der \gls{dl}-basierten \gls{od} beleuchten und aktuelle Trends, Herausforderungen sowie Lösungsansätze identifizieren. Abschließend erfolgt eine systematische Evaluierung der Modellleistung, um die Genauigkeit und Effizienz der Objektdetektion zu bewerten. Dies soll zu optimalen Strategien für die industrielle Qualitätskontrolle beitragen.
