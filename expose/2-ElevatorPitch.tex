\chapter{Elevator Pitch}\label{sec:exp_elevator}
In dieser Bachelorarbeit wird der Einsatz von Deep-Learning-Methoden, speziell der U-Net-Architektur, zur automatischen Detektion von annotierten Merkmalen in Bildern untersucht. Durch die Implementierung, Erweiterung und Evaluierung eines U-Net-Modells soll gezeigt werden, wie trotz Herausforderungen wie begrenzter Datensätze eine präzise und effiziente Erkennung relevanter Merkmale erreicht werden kann. Ziel ist es, einen Beitrag zur Verbesserung der automatisierten Bildanalyse in Anwendungsbereichen zu leisten, in denen traditionelle Methoden an ihre Grenzen stoßen und die Verfügbarkeit von Trainingsdaten eingeschränkt ist.














