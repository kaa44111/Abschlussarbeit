\chapter{Zeitlicher Ablauf}\label{sec:exp_ablauf}
\begin{table}[h!]
    \centering
    \begin{tabular}{|p{3cm}|p{3cm}|p{8cm}|}
        \hline
        \textbf{Phase} & \textbf{Zeitraum} & \textbf{Beschreibung} \\
        \hline
        \textbf{Woche 1 – 2} & Literaturrecherche & 
        - Einarbeitung in den Stand der Technik \newline
        - Lesen und Zusammenfassen relevanter Studien zu U-Net und Bildverarbeitung \\
        \hline
        \textbf{Woche 3 – 4} & Datenpräparierung & 
        - Sammeln und Annotieren von Bilddaten \newline
        - Durchführung von Datenaugmentation mit Bildverarbeitungssoftware \\
        \hline
        \textbf{Woche 5 – 6} & Modellimplementierung & 
        - Auswahl geeigneter Softwarebibliotheken, wie \textit{PyTorch} \newline
        - Implementierung der U-Net-Architektur \\
        \hline
        \textbf{Woche 7 – 8} & Modelltraining und -anpassung & 
        - Training des U-Net-Modells auf den vorbereiteten Datensätzen \newline
        - Optimierung durch Hyperparameter-Tuning und Modellanpassung \\
        \hline
        \textbf{Woche 9-10} & Validierung & 
        - Evaluierung der Modellleistung durch Metriken wie Genauigkeit, Präzision und Recall \\
        \hline
        \textbf{Woche 11} & Ergebnisanalyse & 
        - Analyse und Interpretation der Ergebnisse \newline
        - Diskussion der Stärken und Schwächen der Methode \\
        \hline
        \textbf{Woche 12} & Fertigstellung und Abgabe & 
        - Abschluss der Bachelorarbeit \newline
        - Einholen von Feedback \newline
        - Letzte Überarbeitungen und Abgabe der Arbeit \\
        \hline
    \end{tabular}
    \caption{Zeitlicher Ablauf der Bachelorarbeit}
\end{table}