% durch Austauschen dieser Zeilen kann die Sprache des Templates geändert werden
\PassOptionsToPackage{main=ngerman}{babel}
%\PassOptionsToPackage{main=english}{babel}

\documentclass[thesis]{mi-document}

\bachelor % im Falle einer Masterarbeit \master

% Variablen, die für das Deckblatt und Metadaten verwendet werden
\title{Evaluierung von Einsatzmöglichkeiten einer Deep Learning Applikation zur Detektion gelabelter Merkmale in Bildern}
\author{Amina Kasa}
\semester{Sommersemester 2024}
\course{Bachelorarbeit}
\module{Fakultät Informatik und Mathematik}
\dozent{Prof. Dr. Timo Baumann}
\studid{3190755}
\studSemester{Medizinische Informatik}
\phone{0941/133742666} % Optional
\studSubject{Medizinische Informatik}
\firstReviewer{Prof. Dr. Timo Baumann}
\secondReviewer{[Zweitgutachter]}
\advisor{GEFASOFT Automatisierung und Software GmbH}
\address{Klenzestr. 2a, 93051 Regensburg}{} % Optional
\mail{kasaamina@gmail.com}
\studMail{amina.kasa@st.oth-regensburg.de}
\dateHandedIn{[Abgabetermin der Arbeit]}
\keywords{Enter;key;words;here}

\bibliographystyle{apacite}

% Falls Sie die Abkürzung zum Einbinden von Grafiken benutzen möchten. Erläuterung fnden Sie im Abschnitt zu Abbildungen.
\input{misc/config}

\usepackage{emptypage}
\pagenumbering{roman}
\begin{document}
% ---------------------------------------------------------------

 % Die Nummerierung beginnt mit der Titelseite (= Seite 1), soll aber erst ab der ersten Inhaltsseite (Einleitung) angezeigt werden.
    \pagestyle{empty}


% Das Deckblatt erstellen
    \maketitle
 % ---------------------------------------------------------------
 
%\begin{singlespace}
\pagestyle{plain}
\KOMAoptions{parskip=full}
% Erklärung zur Urherberschaft (urheberschaft.tex) anhängen
\addchap*{}
\centering
\LARGE
\textbf{Erklärung zur Bachelorarbeit von}\\

\RaggedRight
\vspace{60pt}

\normalsize
\doublespacing
\begin{tabularx}{\linewidth}{@{}l<{:}>{\RaggedRight\arraybackslash}X@{}}
Name&Kasa\\
Vorname&Amina\\
Studiengang&\getStudSubject\\
\end{tabularx}

\vspace{40pt}

\begin{enumerate}
    \item{Mir ist bekannt, dass dieses Exemplar der Bachelorarbeit als Prüfungsleistung in das 
Eigentum der Ostbayrischen Technischen Hochschule Regensburg übergeht.}
    \item{ Ich erkläre hiermit, dass ich diese Bachelorarbeit selbständig verfasst, noch nicht 
anderweitig für Prüfungszwecke vorgelegt, keine anderen als die angegebenen Quellen 
und Hilfsmittel benutzt sowie wörtliche und sinngemäße Zitate als solche gekennzeichnet 
habe.}
\end{enumerate}


\signature

\newpage
 
% Abstract hinzufügen
    \clearpage
    \doublespacing

    \input{chapters/0-abstract}
    \clearpage 
    %\stepcounter{page}
% ---------------------------------------------------------------
%Inhaltsverzeichnisse  
\pagestyle{plain}
\singlespacing
\addcontentsline{toc}{chapter}{Inhaltsverzeichnis}
\tableofcontents % Optional
\newpage
%\stepcounter{page}

\addcontentsline{toc}{chapter}{Abbildungsverzeichnis}
\listoffigures % Optional
\newpage
%\stepcounter{page}

\addcontentsline{toc}{chapter}{Tabellenverzeichnis}
\listoftables % Optional
\newpage
%\stepcounter{page}

\addcontentsline{toc}{chapter}{Quellcodeverzeichnis}
\lstlistoflistings % Optional
\newpage
% --------------------------------------------------------------

\pagenumbering{arabic}
\pagestyle{headings} % Seitennummern und Kapitelbezeichnungen anzeigen
    
    % hier beginnt der eigentliche Inhalt der Arbeit
    \chapter{Einleitung}\label{sec:Einleitung}
\pagestyle{headings} % Seitennummern und Kapitelbezeichnungen anzeigen
(30 Literaturquellen ins gesamt.)
Dieses Dokument soll für die Gestaltung von wissenschaftlichen Arbeiten wie Seminararbeiten, Projektdokumentationen, Bachelorarbeiten oder Masterarbeiten am Lehrstuhl für Medieninformatik dienen. Es kann direkt als Word-Vorlage oder nur als Referenz zur Formatierung mit anderen Textsatzprogrammen verwendet werden. 

Ausgehend von den Zielen der Vorlage, werden empfehlenswerte Lehrbücher zum Thema sozusagen als Stand der Technik vorgestellt. Danach werden im Punkt „Gestaltungsrichtlinien“ die Vorgaben für inhaltliche und formale Gestaltung, sowie Zitierweise erläutert. In einem weiteren Abschnitt finden sich Literaturhinweise für empirische Arbeiten sowie Tipps für die Darstellung der Ergebnisse.

Diese Richtlinien dürfen gerne ganz oder teilweise als Grundlage für eigene Richtlinien anderer Lehrstühle verwendet werden. Als Quellenangabe kann „Richtlinien zur Gestaltung schriftlicher Arbeiten, Lehrstuhl für Medieninformatik der Universität Regensburg, Version X.X“ verwendet werden.

\section{Hintergrund und Motivation}

\section{Zielsetzung und Fragestellung}

Die formalen und inhaltlichen Gestaltungsrichtlinien der Dokumentvorlage sollen folgenden Zielen dienen:

\begin{enumerate}
    \item{Die Vorlage soll die Erstellung formal und inhaltlich korrekter Arbeiten mit Word erleichtern.}
    \item{Die formulierten Richtlinien können als Referenz für die Erstellung von wissenschaftlichen Arbeiten mit anderen Textverarbeitungsprogrammen dienen.}
\end{enumerate}

\section{Aufbau der Arbeit}

\section{Grundlagen}

\subsection{Qualitätssicherung}
Die fortschreitende Digitalisierung und Automatisierung in der industriellen Produktion hat die industrielle Bildverarbeitung zu einer Schlüsseltechnologie gemacht. Als Teilgebiet des Maschinellen Sehens (Computer Vision) ermöglicht sie es Maschinen und Systemen, visuelle Informationen aus ihrer Umgebung zu erfassen und zu interpretieren. Dadurch können Aufgaben wie Qualitätskontrolle, Inspektion, Positionierung und Vermessung durchgeführt werden. So werden Produktionsprozesse optimiert, Fehler frühzeitig erkannt und die Effizienz gesteigert \cite{cognex_grundlagen_nodate}.

In der industriellen Sichtprüfung wird überprüft, ob ein Prüfteil den spezifischen Vorgaben entspricht, beispielsweise hinsichtlich Abmessungen, Formen oder der Anwesenheit bestimmter Merkmale. Dieser Prozess umfasst mehrere aufeinanderfolgende Schritte, beginnend mit der Bildaufnahme und -vorverarbeitung, über die Identifikation relevanter Bildbereiche bis hin zur Analyse und Bewertung der Objekteigenschaften \cite[S. 15–16]{demant_industrielle_2011}.

Für ein umfassendes Bildverständnis ist es nach der Erstellung eines digitalen Bildes wichtig, nicht nur das Vorhandensein bestimmter Merkmale zu erkennen, sondern auch deren genaue Position und Ausdehnung im Bild zu bestimmen. Diese Aufgabe wird als Objektdetektion definiert. Sie liefert wertvolle Informationen für das semantische Verständnis von Bildern und Videos und findet Anwendung in Bereichen wie Bildklassifikation, Verhaltensanalyse, Gesichtserkennung und autonomem Fahren \cite{zhao_object_2019}. Dennoch können Eigenschaften wie spezielle Texturen, Formen, Farben oder Größen von Objekten die zuverlässige Detektion erschweren, da Fehler mit dem Hintergrund verschmelzen und somit eine präzise Analyse verhindern.

\subsection{Deep Learning}
Convolutional Neural Networks (CNNs) haben sich hierbei als effektive Methode erwiesen, um Maschinen das "Sehen" beizubringen und komplexe Muster in Bildern zu erkennen. Sie haben sich in der Bildverarbeitung etabliert, insbesondere bei Aufgaben der Bildklassifikation, Objekterkennung und Bildsegmentierung \cite{qi_review_2020}. (Ergnzen nur hier!!!)

Ziel dieser Arbeit ist es, eine Deep-Learning-Anwendung zu entwickeln und zu evaluieren, die bestimmte gekennzeichnete Merkmale in Bildern automatisch erkennen kann. Dabei wird untersucht, wie mithilfe von CNNs und der semantischen Segmentierung trotz Herausforderungen wie begrenzter Datenmenge und Bildrauschen eine zuverlässige Erkennung möglich ist.

 In den vergangenen Jahren hat sich gezeigt, dass Deep-Learning-Modelle, insbesondere Convolutional Neural Networks (CNNs), eine neue Ära der Bildverarbeitung eingeleitet haben. Sie erzielen signifikant bessere Ergebnisse bei Aufgaben wie der Bildklassifikation, Objekterkennung und Segmentierung, was zu einer deutlichen Leistungssteigerung in verschiedenen Anwendungsbereichen geführt hat.\cite{minaee_image_2022}

Ein CNN besteht aus mehreren Schichten, von denen jede spezifische Funktionen auf die Eingabedaten anwendet. Zu den Hauptkomponenten gehören die Eingabeschicht, konvolutionale Schichten, Aktivierungsschichten, Pooling-Schichten und vollständig verbundene Schichten. Diese Architektur ermöglicht es CNNs, komplexe Merkmale aus Bildern zu extrahieren und zu interpretieren.\cite{jogin_feature_2018}

    \chapter{Theoretische Grundlagen und Stand der Technik}\label{sec:stand_der_technik}

\section{Industrielle Bildverarbeitung}
 (Vielleicht noch die traditionelle Methoden, Maschine Learnig und Deep Learning erklären)
 
\section{Qualitätssicherung}
Die fortschreitende Digitalisierung und Automatisierung in der industriellen Produktion hat die industrielle Bildverarbeitung zu einer Schlüsseltechnologie gemacht. Als Teilgebiet des Maschinellen Sehens (Computer Vision) ermöglicht sie es Maschinen und Systemen, visuelle Informationen aus ihrer Umgebung zu erfassen und zu interpretieren. Dadurch können Aufgaben wie Qualitätskontrolle, Inspektion, Positionierung und Vermessung durchgeführt werden. So werden Produktionsprozesse optimiert, Fehler frühzeitig erkannt und die Effizienz gesteigert \cite{cognex_grundlagen_nodate}.

In der industriellen Sichtprüfung wird überprüft, ob ein Prüfteil den spezifischen Vorgaben entspricht, beispielsweise hinsichtlich Abmessungen, Formen oder der Anwesenheit bestimmter Merkmale. Dieser Prozess umfasst mehrere aufeinanderfolgende Schritte, beginnend mit der Bildaufnahme und -vorverarbeitung, über die Identifikation relevanter Bildbereiche bis hin zur Analyse und Bewertung der Objekteigenschaften \cite[S. 15–16]{demant_industrielle_2011}.

Für ein umfassendes Bildverständnis ist es nach der Erstellung eines digitalen Bildes wichtig, nicht nur das Vorhandensein bestimmter Merkmale zu erkennen, sondern auch deren genaue Position und Ausdehnung im Bild zu bestimmen. Diese Aufgabe wird als Objektdetektion definiert. Sie liefert wertvolle Informationen für das semantische Verständnis von Bildern und Videos und findet Anwendung in Bereichen wie Bildklassifikation, Verhaltensanalyse, Gesichtserkennung und autonomem Fahren \cite{zhao_object_2019}. Dennoch können Eigenschaften wie spezielle Texturen, Formen, Farben oder Größen von Objekten die zuverlässige Detektion erschweren, da Fehler mit dem Hintergrund verschmelzen und somit eine präzise Analyse verhindern.

\section{Deep Learning}

Convolutional Neural Networks (CNNs) haben sich hierbei als effektive Methode erwiesen, um Maschinen das "Sehen" beizubringen und komplexe Muster in Bildern zu erkennen. Sie haben sich in der Bildverarbeitung etabliert, insbesondere bei Aufgaben der Bildklassifikation, Objekterkennung und Bildsegmentierung \cite{qi_review_2020}. (Ergnzen nur hier!!!)

Ziel dieser Arbeit ist es, eine Deep-Learning-Anwendung zu entwickeln und zu evaluieren, die bestimmte gekennzeichnete Merkmale in Bildern automatisch erkennen kann. Dabei wird untersucht, wie mithilfe von CNNs und der semantischen Segmentierung trotz Herausforderungen wie begrenzter Datenmenge und Bildrauschen eine zuverlässige Erkennung möglich ist.

 In den vergangenen Jahren hat sich gezeigt, dass Deep-Learning-Modelle, insbesondere Convolutional Neural Networks (CNNs), eine neue Ära der Bildverarbeitung eingeleitet haben. Sie erzielen signifikant bessere Ergebnisse bei Aufgaben wie der Bildklassifikation, Objekterkennung und Segmentierung, was zu einer deutlichen Leistungssteigerung in verschiedenen Anwendungsbereichen geführt hat.\cite{minaee_image_2022}

Ein CNN besteht aus mehreren Schichten, von denen jede spezifische Funktionen auf die Eingabedaten anwendet. Zu den Hauptkomponenten gehören die Eingabeschicht, konvolutionale Schichten, Aktivierungsschichten, Pooling-Schichten und vollständig verbundene Schichten. Diese Architektur ermöglicht es CNNs, komplexe Merkmale aus Bildern zu extrahieren und zu interpretieren.\cite{jogin_feature_2018}

\subsection{Supervised Learning}
\subsection{Unsupervised Learning}
\subsection{Das U-Net Modell}
\section{Aktuelle Forschung und Entwicklungen}

    \include{chapters/3-bilddaten}
    \chapter{Implementierung und Experimente}\label{sec:model_implement}

\section{Präparierung der Bilddaten}
\subsection{Vorbereitung und Beschreibung des Datensatzes}
\subsection{Labeling-Prozess}
\subsection{Datenaugmentationstechniken}

\section{Modellierung und Training}
\subsection{Auswahl der Softwarebibliotheken} 
\subsection{Modellarchitektur}
\subsection{Trainingsprozess und Hyperparameter-Optimierung}
\section{Validierung und Bewertung der Ergebnisse}
\subsection{Methoden der Validierung}
\subsection{Bewertungsmethoden und Metriken}
\subsection{Analyse und Diskussion der Ergebnisse}

    \chapter{Validierung und Bewertng der Ergebnisse}\label{sec:validierung}
[Text]

\section{Methoden der Validierung}

\section{Bewertungsmethoden und Metriken}

\section{Analyse und Diskussion der Ergebnisse}

    \include{chapters/6-fazit_ausblick}
    
    % Kapitelbezeichnung in der rechten oberen Ecke entfernen
\clearpage
\pagestyle{plain}
    
% Literaturverzeichnis anzeigen
% kleinerer Zeilenabstand, damit es nicht so gestreckt aussieht
\onehalfspacing
\bibliography{references}
\doublespacing

% Anhang
\appendix
\include{misc/anhang}

\begin{singlespace}
\KOMAoptions{parskip=full}

% Erklärung zur Lizenzierung der Arbeit (lizenzierung.tex) anhängen
\include{misc/lizenzierung}

% Stichwortverzeichnis anzeigen. Weiß nicht, warum das nicht nach dem Inhaltsverzeichnis kommt.
\printindex
\end{singlespace}

% Inhalt des Datenträgers. Gehört meiner Meinung nach zum Anhang, aber was weiß ich schon. AS
\newpage
\input{misc/datenträger}

\end{document}
