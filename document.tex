% durch Austauschen dieser Zeilen kann die Sprache des Templates geändert werden
\PassOptionsToPackage{main=ngerman}{babel}
%\PassOptionsToPackage{main=english}{babel}

\documentclass[thesis]{mi-document}

\bachelor % im Falle einer Masterarbeit \master

% Variablen, die für das Deckblatt und Metadaten verwendet werden
\title{Evaluierung von Einsatzmöglichkeiten einer Deep Learning Applikation zur Detektion gelabelter Merkmale in Bildern}
\author{Amina Kasa}
\semester{Sommersemester 2024}
\course{Bachelorarbeit}
\module{Fakultät Informatik und Mathematik}
\dozent{Prof. Dr. Timo Baumann}
\studid{3190755}
\studSemester{Medizinische Informatik}
\phone{0941/133742666} % Optional
\studSubject{Medizinische Informatik}
\firstReviewer{Prof. Dr. Timo Baumann}
\secondReviewer{[Zweitgutachter]}
\advisor{GEFASOFT Automatisierung und Software GmbH}
\address{Klenzestr. 2a, 93051 Regensburg}{} % Optional
\mail{kasaamina@gmail.com}
\studMail{amina.kasa@st.oth-regensburg.de}
\dateHandedIn{[Abgabetermin der Arbeit]}
\keywords{Enter;key;words;here}

\bibliographystyle{apacite}

% Falls Sie die Abkürzung zum Einbinden von Grafiken benutzen möchten. Erläuterung fnden Sie im Abschnitt zu Abbildungen.
\newcommand*{\image}[2]{
	\begin{figure}
	\centering
	\includegraphics[width=0.5\textwidth]{{images/#1}}
	\caption{#2}
	\label{fig:#1}
	\end{figure}
}

\usepackage{emptypage}
\pagenumbering{roman}
\begin{document}
% ---------------------------------------------------------------

 % Die Nummerierung beginnt mit der Titelseite (= Seite 1), soll aber erst ab der ersten Inhaltsseite (Einleitung) angezeigt werden.
    \pagestyle{empty}


% Das Deckblatt erstellen
    \maketitle
 % ---------------------------------------------------------------
 
%\begin{singlespace}
\pagestyle{plain}
\KOMAoptions{parskip=full}
% Erklärung zur Urherberschaft (urheberschaft.tex) anhängen
\addchap*{}
\centering
\LARGE
\textbf{Erklärung zur Bachelorarbeit von}\\

\RaggedRight
\vspace{60pt}

\normalsize
\doublespacing
\begin{tabularx}{\linewidth}{@{}l<{:}>{\RaggedRight\arraybackslash}X@{}}
Name&Kasa\\
Vorname&Amina\\
Studiengang&\getStudSubject\\
\end{tabularx}

\vspace{40pt}

\begin{enumerate}
    \item{Mir ist bekannt, dass dieses Exemplar der Bachelorarbeit als Prüfungsleistung in das 
Eigentum der Ostbayrischen Technischen Hochschule Regensburg übergeht.}
    \item{ Ich erkläre hiermit, dass ich diese Bachelorarbeit selbständig verfasst, noch nicht 
anderweitig für Prüfungszwecke vorgelegt, keine anderen als die angegebenen Quellen 
und Hilfsmittel benutzt sowie wörtliche und sinngemäße Zitate als solche gekennzeichnet 
habe.}
\end{enumerate}


\signature

\newpage
 
% Abstract hinzufügen
    \clearpage
    \doublespacing

    \begin{abstract}[ngerman]
Bachelor- und Masterarbeiten beginnen mit einer Zusammenfassung in einer deutschen und englischen Version.
Die Zusammenfassung gibt einen Überblick über Thema und Resultate der Arbeit.
Inhaltlich werden die Zielsetzung, die Methodik, die einzelnen Arbeitsschritte bzw. Gliederungspunkte und die Ergebnisse der Arbeit widergegeben.

Schlecht: „Schon immer haben Menschen Zusammenfassungen geschrieben [Platitüde, keine Zusammenfassung]. In dieser Arbeit wurde in mehreren Studien untersucht, wie Zusammenfassungen wirken. [Was genau wurde untersucht? Was waren die Ergebnisse?]“

Besser: „In dieser Arbeit wurde untersucht, inwiefern das Lesen von Zusammenfassungen das Lesen des kompletten Dokuments ersetzen kann. Dazu wurden zwei Studien mit jeweils 17 Teilnehmern durchgeführt. In der ersten wurde [...]. Diese Ergebnisse zeigen, dass Bedienungsanleitungen und Bilderbücher weniger gut über Zusammenfassungen erschlossen werden können, als Romane oder Sachbücher.“
\end{abstract}

\begin{abstract}[english]
A summary in English. It should be more or less similar to the German Zusammenfassung. Avoid too verbatim translations („In this work it was examined how the reading of ...“)
\end{abstract}

    \clearpage 
    %\stepcounter{page}
% ---------------------------------------------------------------
%Inhaltsverzeichnisse  
\pagestyle{plain}
\singlespacing
\addcontentsline{toc}{chapter}{Inhaltsverzeichnis}
\tableofcontents % Optional
\newpage
%\stepcounter{page}

\addcontentsline{toc}{chapter}{Abbildungsverzeichnis}
\listoffigures % Optional
\newpage
%\stepcounter{page}

\addcontentsline{toc}{chapter}{Tabellenverzeichnis}
\listoftables % Optional
\newpage
%\stepcounter{page}

\addcontentsline{toc}{chapter}{Quellcodeverzeichnis}
\lstlistoflistings % Optional
\newpage
% --------------------------------------------------------------

\pagenumbering{arabic}
\pagestyle{headings} % Seitennummern und Kapitelbezeichnungen anzeigen
    
    % hier beginnt der eigentliche Inhalt der Arbeit
    \chapter{Einleitung}\label{sec:Einleitung}
\pagestyle{headings} % Seitennummern und Kapitelbezeichnungen anzeigen
Dieses Dokument soll für die Gestaltung von wissenschaftlichen Arbeiten wie Seminararbeiten, Projektdokumentationen, Bachelorarbeiten oder Masterarbeiten am Lehrstuhl für Medieninformatik dienen. Es kann direkt als Word-Vorlage oder nur als Referenz zur Formatierung mit anderen Textsatzprogrammen verwendet werden. 

Ausgehend von den Zielen der Vorlage, werden empfehlenswerte Lehrbücher zum Thema sozusagen als Stand der Technik vorgestellt. Danach werden im Punkt „Gestaltungsrichtlinien“ die Vorgaben für inhaltliche und formale Gestaltung, sowie Zitierweise erläutert. In einem weiteren Abschnitt finden sich Literaturhinweise für empirische Arbeiten sowie Tipps für die Darstellung der Ergebnisse.

Diese Richtlinien dürfen gerne ganz oder teilweise als Grundlage für eigene Richtlinien anderer Lehrstühle verwendet werden. Als Quellenangabe kann „Richtlinien zur Gestaltung schriftlicher Arbeiten, Lehrstuhl für Medieninformatik der Universität Regensburg, Version X.X“ verwendet werden.

\section{Hintergrund und Motivation}

\section{Zielsetzung und Fragestellung}

\section{Aufbau der Arbeit}

    \chapter{Literaturrecherche und Stand der Technik}\label{sec:stand_der_technik}

Es gibt zahlreiche Ratgeber\index{Ratgeber} für das wissenschaftliche Arbeiten und Schreiben. Die Handbücher unterscheiden sich in inhaltlichen Schwerpunkt, praktischer Orientierung und Vertiefung der einzelnen Themen. Drei sehr empfehlenswerte Ratgeber sollen kurz vorgestellt werden.

\cite{karmasin2012gestaltung} bieten einen sehr knappen und praktisch orientierten Ratgeber. Es werden inhaltliche und formale Anforderungen an wissenschaftliche Arbeiten wie inhaltlicher Aufbau der Kapitel, Bewertungskriterien und formale Aspekte wie Gliederung behandelt. Daneben enthält der Ratgeber ein eigenes Kapitel mit Tipps zur Formatierung mit Word.

Das Handbuch von \cite{esselborn2012richtig} konzentriert sich auf die Frage nach dem richtigen wissenschaftlichen Sprachstil. Es werden konkrete Regeln und Übungen vorgestellt um sprachliche Präzision und gedankliche Klarheit im Text zu erreichen. Daneben wird in einem eigenen Kapitel auf die häufigsten Fehler beim wissenschaftlichen Schreiben hingewiesen.

\cite{balzert2011wissenschaftliches} bieten einen sehr ausführlichen Ratgeber zum wissenschaftlichen Arbeiten. Im ersten Teil werden Qualitätskriterien und Methoden als Grundlagen wissenschaftlicher Arbeit aufgezeigt. Im zweiten Teil werden verschiedene wissenschaftliche Artefakte also Textformen gegenübergestellt und der formale Aufbau wissenschaftlicher Arbeiten beleuchtet. Im dritten Teil werden Empfehlungen zum Erstellungsprozess einer Arbeit mit Projektplan etc. gegeben. Im letzten Teil werden verschiedene Aspekte der Präsentation behandelt, wie z. B. Vortragsformen mit und ohne visuelle Unterstützung oder der richtige Vortragsstil.

\section{Überblick über die Industrielle Bildverarbeitung}

\section{Deep Learning in der Bildverarbeitung}

\section{Aktuelle Forschung und Entwicklungen}


    \chapter{Präparierung der Bilddaten}\label{sec:Bilddaten}
[Text]

\section{Beschreibung des Datensatzes}

\section{Labeling-Prozess}

\section{Datenaugumentationstechniken}


    \chapter{Modellierung und Training}\label{sec:model_train}
[Text]

\section{Auswahl der Softwarebibliotheken}

\section{Modellarchitektur}

\section{Trainingsprozess und Hyperparameter-Optimierung}

    \chapter{Validierung und Bewertng der Ergebnisse}\label{sec:validierung}
[Text]

\section{Methoden der Validierung}

\section{Bewertungsmethoden und Metriken}

\section{Analyse und Diskussion der Ergebnisse}

    \chapter{Fazit und Ausblick}\label{sec:Fazit}

Die hier vorliegende Dokumentvorlage soll die Gestaltung wissenschaftlicher Arbeiten am Lehrstuhl für Medieninformatik erleichtern und zur Qualitätssicherung beitragen. Es werden Richtlinien für den inhaltlichen Aufbau und die formale Gestaltung formuliert. Dabei ist die Vorlage selbst wie eine wissenschaftliche Arbeit strukturiert und enthält die wichtigsten Formatvorlagen zur effizienten Gestaltung mit Word. Als zusätzliches Hilfsmittel für den inhaltlichen Aufbau verschiedener Arbeiten befinden sich im Anhang typische Bausteine bzw. Mustergliederungen für verschiedene Thementypen, wie „theoretische“, „konstruktive“ oder „empirische“ Arbeiten. Für weitere Informationen sind im Kapitel \ref{sec:stand_der_technik} verschiedene Ratgeber kurz vorgestellt. 

\section{Zusammenfassung der Ergebnisse}

\section{Diskussion der Ergebnisse}

\section{Ausblick auf Zukünftige Arbeiten}

    
    % Kapitelbezeichnung in der rechten oberen Ecke entfernen
\clearpage
\pagestyle{plain}
    
% Literaturverzeichnis anzeigen
% kleinerer Zeilenabstand, damit es nicht so gestreckt aussieht
\onehalfspacing
\bibliography{literature}
\doublespacing

% Anhang
\appendix
\chapter{Bausteine wissenschaftlicher Arbeiten}\label{sec:anhang}\index{Bausteine wiss. Arbeiten}

\section{Theoretische Arbeit}\label{subsec:a1}

\begin{enumerate}
    \item{Fragestellung (Ziele, Motivation)}
    \item{Überblick über Stand der Forschung und Technik (dabei Bewertung der Ansätze, Beispiele, Identifikation von Defiziten)}
    \item{Synthese: Erstellung einer Gesamtschau (allgemeine Prinzipien, Beschreibung einer eigenen Sicht auf das Problem, Formulierung von Empfehlungen )}
    \item{Zusammenfassung (Was wurde in der Arbeit erreicht, Erklärung des Nutzens für andere)}
    \item{Ausblick (optional)}
\end{enumerate}

\section{Konstruktive Arbeit}\label{subsec:a2}

\begin{enumerate}
    \item{Problemstellung (Ziele, Ausgangspunkt, Vorgesehener Benutzerkreis, Bedürfnisse der Benutzer)}
    \item{Stand der Forschung und Technik (Bisherige Lösungen, Defizite)}
    \item{Eigenes Konzept (Lösungsansatz, allgemeines Prinzip, Werkzeuge z.B. Programmiersprachen )}
    \item{Vorgehensweise (Beschreibung der durchgeführten Arbeitsschritte)}
    \item{Ergebnis (Vorstellung des System z.B. Screenshots mit Erläuterungen)}
    \item{Evaluation des System (optional, was soll evaluiert werden, welche Methode, Ablauf, Ergebnisse)}
    \item{Zusammenfassung (Was wurde in der Arbeit erreicht; Erklärung des Nutzens für andere)}
    \item{Ausblick (optional)}
\end{enumerate}

\section{Empirische Arbeit}\label{subsec:a3}

\begin{enumerate}
    \item{Fragestellung der Arbeit (Was soll untersucht werden, warum)}
    \item{Stand der Forschung und Technik (Bewertung der Untersuchungs-Ansätze und Ergebnisse, Identifikation von Defiziten)}
    \item{Präzisierung der Fragestellung (Hypothesen)}
    \item{Untersuchungsmethodik }
    \item{Untersuchungsablauf (Untersuchungsmaterial, Raum, Probandenrekrutierung etc.)}
    \item{Ergebnisse (Darstellung der Ergebnisse in sinnvoller  Reihenfolge, Gesamtüberblick, Einzelergebnisse z. B. geordnet nach Testcases)}
    \item{Zusammenfassung (Was wurde erreicht, Rückbezug zu Zielen, Hypothesen, Nutzen, Erkenntnisse für weitere Untersuchungen)}
    \item{Ausblick (optional)}
\end{enumerate}



\begin{singlespace}
\KOMAoptions{parskip=full}

% Erklärung zur Lizenzierung der Arbeit (lizenzierung.tex) anhängen
\addchap{Erklärung zur Lizenzierung und Publikation dieser Arbeit}

\textbf{Name:} \getAuthor

\textbf{Titel der Arbeit:} \textit{\getTitle}

Hiermit gestatte ich die Verwendung der schriftlichen Ausarbeitung zeitlich unbegrenzt und nicht-exklusiv unter folgenden Bedingungen:

\begin{itemize}
    \item[\checkboxEmpty] Nur zur Bewertung dieser Arbeit
    \item[\checkboxEmpty] Nur innerhalb des Lehrstuhls im Rahmen von Forschung und Lehre
    \item[\checkboxChecked] Unter einer Creative-Commons-Lizenz mit den folgenden Einschränkungen:
    \begin{itemize}
        \item[\checkboxChecked] BY – Namensnennung des Autors
        \item[\checkboxEmpty] NC – Nichtkommerziell
        \item[\checkboxEmpty] SA – Share-Alike, d.h. alle Änderungen müssen unter die gleiche Lizenz gestellt werden.
    \end{itemize}
\end{itemize}
{\scriptsize(An Zitaten und Abbildungen aus fremden Quellen werden keine weiteren Rechte eingeräumt.)}

Außerdem gestatte ich die Verwendung des im Rahmen dieser Arbeit erstellten Quellcodes unter folgender Lizenz:

\begin{itemize}
    \item[\checkboxEmpty] Nur zur Bewertung dieser Arbeit
    \item[\checkboxEmpty] Nur innerhalb des Lehrstuhls im Rahmen von Forschung und Lehre
    \item[\checkboxEmpty] Unter der CC-0-Lizenz (= beliebige Nutzung)
    \item[\checkboxChecked] Unter der MIT-Lizenz (= Namensnennung)
    \item[\checkboxEmpty] Unter der GPLv3-Lizenz (oder neuere Versionen)
\end{itemize}

{\scriptsize(An explizit mit einer anderen Lizenz gekennzeichneten Bibliotheken und Daten werden keine weiteren Rechte eingeräumt.)}

\noindent
Ich willige ein, dass der Lehrstuhl  für Medieninformatik diese Arbeit – falls sie besonders gut ausfällt - auf dem Publikationsserver der Universität Regensburg veröffentlichen lässt.

\noindent
Ich übertrage deshalb der Universität Regensburg das Recht, die Arbeit elektronisch zu speichern und in Datennetzen öffentlich zugänglich zu machen. Ich übertrage der Universität Regensburg ferner das Recht zur Konvertierung zum Zwecke der Langzeitarchivierung unter Beachtung der Bewahrung des Inhalts (die Originalarchivierung bleibt erhalten).

\noindent
Ich erkläre außerdem, dass von mir die urheber- und lizenzrechtliche Seite (Copyright) geklärt wurde und Rechte Dritter der Publikation nicht entgegenstehen.

\begin{itemize}
    \item[\checkboxChecked] Ja, für die komplette Arbeit inklusive Anhang
    \item[\checkboxEmpty] Ja, für eine um vertrauliche Informationen gekürzte Variante (auf dem Datenträger beigefügt)
    \item[\checkboxEmpty] Nein
    \item[\checkboxEmpty] Sperrvermerk bis (Datum):
    %Sperrvermerke sind mit dem Betreuer am Lehrstuhl abzustimmen. Sperrvermerke mit einer Frist von mehr als zwei Jahren benötigen immer eine schriftliche Begründung, aus der hervorgeht, weshalb eine kürzere Sperrfrist nicht ausreichend ist.
\end{itemize}

\signature


% Stichwortverzeichnis anzeigen. Weiß nicht, warum das nicht nach dem Inhaltsverzeichnis kommt.
\printindex
\end{singlespace}

% Inhalt des Datenträgers. Gehört meiner Meinung nach zum Anhang, aber was weiß ich schon. AS
\newpage
\input{misc/datenträger}

\end{document}
